\documentclass[]{article}
\usepackage{lmodern}
\usepackage{amssymb,amsmath}
\usepackage{ifxetex,ifluatex}
\usepackage{fixltx2e} % provides \textsubscript
\ifnum 0\ifxetex 1\fi\ifluatex 1\fi=0 % if pdftex
  \usepackage[T1]{fontenc}
  \usepackage[utf8]{inputenc}
\else % if luatex or xelatex
  \ifxetex
    \usepackage{mathspec}
  \else
    \usepackage{fontspec}
  \fi
  \defaultfontfeatures{Ligatures=TeX,Scale=MatchLowercase}
\fi
% use upquote if available, for straight quotes in verbatim environments
\IfFileExists{upquote.sty}{\usepackage{upquote}}{}
% use microtype if available
\IfFileExists{microtype.sty}{%
\usepackage{microtype}
\UseMicrotypeSet[protrusion]{basicmath} % disable protrusion for tt fonts
}{}
\usepackage[margin=1in]{geometry}
\usepackage{hyperref}
\hypersetup{unicode=true,
            pdftitle={GeoPhy Task},
            pdfauthor={Nirupama Benis},
            pdfborder={0 0 0},
            breaklinks=true}
\urlstyle{same}  % don't use monospace font for urls
\usepackage{color}
\usepackage{fancyvrb}
\newcommand{\VerbBar}{|}
\newcommand{\VERB}{\Verb[commandchars=\\\{\}]}
\DefineVerbatimEnvironment{Highlighting}{Verbatim}{commandchars=\\\{\}}
% Add ',fontsize=\small' for more characters per line
\usepackage{framed}
\definecolor{shadecolor}{RGB}{248,248,248}
\newenvironment{Shaded}{\begin{snugshade}}{\end{snugshade}}
\newcommand{\KeywordTok}[1]{\textcolor[rgb]{0.13,0.29,0.53}{\textbf{#1}}}
\newcommand{\DataTypeTok}[1]{\textcolor[rgb]{0.13,0.29,0.53}{#1}}
\newcommand{\DecValTok}[1]{\textcolor[rgb]{0.00,0.00,0.81}{#1}}
\newcommand{\BaseNTok}[1]{\textcolor[rgb]{0.00,0.00,0.81}{#1}}
\newcommand{\FloatTok}[1]{\textcolor[rgb]{0.00,0.00,0.81}{#1}}
\newcommand{\ConstantTok}[1]{\textcolor[rgb]{0.00,0.00,0.00}{#1}}
\newcommand{\CharTok}[1]{\textcolor[rgb]{0.31,0.60,0.02}{#1}}
\newcommand{\SpecialCharTok}[1]{\textcolor[rgb]{0.00,0.00,0.00}{#1}}
\newcommand{\StringTok}[1]{\textcolor[rgb]{0.31,0.60,0.02}{#1}}
\newcommand{\VerbatimStringTok}[1]{\textcolor[rgb]{0.31,0.60,0.02}{#1}}
\newcommand{\SpecialStringTok}[1]{\textcolor[rgb]{0.31,0.60,0.02}{#1}}
\newcommand{\ImportTok}[1]{#1}
\newcommand{\CommentTok}[1]{\textcolor[rgb]{0.56,0.35,0.01}{\textit{#1}}}
\newcommand{\DocumentationTok}[1]{\textcolor[rgb]{0.56,0.35,0.01}{\textbf{\textit{#1}}}}
\newcommand{\AnnotationTok}[1]{\textcolor[rgb]{0.56,0.35,0.01}{\textbf{\textit{#1}}}}
\newcommand{\CommentVarTok}[1]{\textcolor[rgb]{0.56,0.35,0.01}{\textbf{\textit{#1}}}}
\newcommand{\OtherTok}[1]{\textcolor[rgb]{0.56,0.35,0.01}{#1}}
\newcommand{\FunctionTok}[1]{\textcolor[rgb]{0.00,0.00,0.00}{#1}}
\newcommand{\VariableTok}[1]{\textcolor[rgb]{0.00,0.00,0.00}{#1}}
\newcommand{\ControlFlowTok}[1]{\textcolor[rgb]{0.13,0.29,0.53}{\textbf{#1}}}
\newcommand{\OperatorTok}[1]{\textcolor[rgb]{0.81,0.36,0.00}{\textbf{#1}}}
\newcommand{\BuiltInTok}[1]{#1}
\newcommand{\ExtensionTok}[1]{#1}
\newcommand{\PreprocessorTok}[1]{\textcolor[rgb]{0.56,0.35,0.01}{\textit{#1}}}
\newcommand{\AttributeTok}[1]{\textcolor[rgb]{0.77,0.63,0.00}{#1}}
\newcommand{\RegionMarkerTok}[1]{#1}
\newcommand{\InformationTok}[1]{\textcolor[rgb]{0.56,0.35,0.01}{\textbf{\textit{#1}}}}
\newcommand{\WarningTok}[1]{\textcolor[rgb]{0.56,0.35,0.01}{\textbf{\textit{#1}}}}
\newcommand{\AlertTok}[1]{\textcolor[rgb]{0.94,0.16,0.16}{#1}}
\newcommand{\ErrorTok}[1]{\textcolor[rgb]{0.64,0.00,0.00}{\textbf{#1}}}
\newcommand{\NormalTok}[1]{#1}
\usepackage{graphicx,grffile}
\makeatletter
\def\maxwidth{\ifdim\Gin@nat@width>\linewidth\linewidth\else\Gin@nat@width\fi}
\def\maxheight{\ifdim\Gin@nat@height>\textheight\textheight\else\Gin@nat@height\fi}
\makeatother
% Scale images if necessary, so that they will not overflow the page
% margins by default, and it is still possible to overwrite the defaults
% using explicit options in \includegraphics[width, height, ...]{}
\setkeys{Gin}{width=\maxwidth,height=\maxheight,keepaspectratio}
\IfFileExists{parskip.sty}{%
\usepackage{parskip}
}{% else
\setlength{\parindent}{0pt}
\setlength{\parskip}{6pt plus 2pt minus 1pt}
}
\setlength{\emergencystretch}{3em}  % prevent overfull lines
\providecommand{\tightlist}{%
  \setlength{\itemsep}{0pt}\setlength{\parskip}{0pt}}
\setcounter{secnumdepth}{0}
% Redefines (sub)paragraphs to behave more like sections
\ifx\paragraph\undefined\else
\let\oldparagraph\paragraph
\renewcommand{\paragraph}[1]{\oldparagraph{#1}\mbox{}}
\fi
\ifx\subparagraph\undefined\else
\let\oldsubparagraph\subparagraph
\renewcommand{\subparagraph}[1]{\oldsubparagraph{#1}\mbox{}}
\fi

%%% Use protect on footnotes to avoid problems with footnotes in titles
\let\rmarkdownfootnote\footnote%
\def\footnote{\protect\rmarkdownfootnote}

%%% Change title format to be more compact
\usepackage{titling}

% Create subtitle command for use in maketitle
\newcommand{\subtitle}[1]{
  \posttitle{
    \begin{center}\large#1\end{center}
    }
}

\setlength{\droptitle}{-2em}
  \title{GeoPhy Task}
  \pretitle{\vspace{\droptitle}\centering\huge}
  \posttitle{\par}
  \author{Nirupama Benis}
  \preauthor{\centering\large\emph}
  \postauthor{\par}
  \predate{\centering\large\emph}
  \postdate{\par}
  \date{April 22, 2018}


\begin{document}
\maketitle

\subsection{Overview of task}\label{overview-of-task}

The assigned task was to download two datasets from the New York City
Open Data website. One dataset contains information on the condomonidum
rent income and the other has information on New York City community
gardens. This script shows how I downloaded the datasets and processed
them to answer the given questions.

\subsection{Datasets}\label{datasets}

I downloaded the CSV versions of the datasets, although other versions
like the RDF were available I did not see an added benefit in working
with it for this task. Then I loaded the datasets into the R
environment.

\begin{Shaded}
\begin{Highlighting}[]
\NormalTok{condoDataset <-}\StringTok{ }\KeywordTok{read.csv}\NormalTok{(}\DataTypeTok{file =} \StringTok{"/home/niru/allInfoDir/otherWorkspaces/geophy/Data/DOF__Condominium_comparable_rental_income___Manhattan_-_FY_2010_2011.csv"}\NormalTok{, }\DataTypeTok{as.is =}\NormalTok{ T)}
\NormalTok{condoDataset <-}\StringTok{ }\NormalTok{condoDataset[}\OperatorTok{!}\KeywordTok{is.na}\NormalTok{(condoDataset}\OperatorTok{$}\NormalTok{Latitude), ]}

\NormalTok{greenThumbDataset <-}\StringTok{ }\KeywordTok{read.csv}\NormalTok{(}\DataTypeTok{file =} \StringTok{"/home/niru/allInfoDir/otherWorkspaces/geophy/Data/NYC_Greenthumb_Community_Gardens.csv"}\NormalTok{, }\DataTypeTok{as.is =}\NormalTok{ T)}
\NormalTok{greenThumbDataset <-}\StringTok{ }\NormalTok{greenThumbDataset[}\OperatorTok{!}\KeywordTok{is.na}\NormalTok{(greenThumbDataset}\OperatorTok{$}\NormalTok{Latitude), ]}
\end{Highlighting}
\end{Shaded}

\subsection{Task 1 - Visualize the datasets on a
map}\label{task-1---visualize-the-datasets-on-a-map}

I used the R package \texttt{ggmaps} to plot the data points

\begin{Shaded}
\begin{Highlighting}[]
\KeywordTok{library}\NormalTok{(ggmap)}
\end{Highlighting}
\end{Shaded}

\begin{verbatim}
## Loading required package: ggplot2
\end{verbatim}

The Green Thumbs dataset has a bigger geographical spread than the
condomonium dataset so I used that the center for the map of New York
City.

\begin{Shaded}
\begin{Highlighting}[]
\NormalTok{centerGreen <-}\StringTok{ }\KeywordTok{c}\NormalTok{(}\KeywordTok{mean}\NormalTok{(}\KeywordTok{range}\NormalTok{(greenThumbDataset}\OperatorTok{$}\NormalTok{Longitude)), }\KeywordTok{mean}\NormalTok{(}\KeywordTok{range}\NormalTok{(greenThumbDataset}\OperatorTok{$}\NormalTok{Latitude)))}
\NormalTok{mapNY <-}\StringTok{ }\KeywordTok{get_googlemap}\NormalTok{(}\DataTypeTok{center =}\NormalTok{ centerGreen, }\DataTypeTok{zoom =} \DecValTok{11}\NormalTok{)}

\KeywordTok{ggmap}\NormalTok{(mapNY)}
\end{Highlighting}
\end{Shaded}

\includegraphics{geoPhyTask_files/figure-latex/visualize2-1.pdf}

Then I plotted the condomoniums (in black) and the community gardens (in
purple).

\begin{Shaded}
\begin{Highlighting}[]
\NormalTok{mapPoints <-}\StringTok{ }\KeywordTok{ggmap}\NormalTok{(mapNY) }\OperatorTok{+}\StringTok{ }
\StringTok{  }\KeywordTok{geom_point}\NormalTok{(}\KeywordTok{aes}\NormalTok{(}\DataTypeTok{x =}\NormalTok{ condoDataset}\OperatorTok{$}\NormalTok{Longitude, }\DataTypeTok{y =}\NormalTok{ condoDataset}\OperatorTok{$}\NormalTok{Latitude), }\DataTypeTok{data =}\NormalTok{ condoDataset, }\DataTypeTok{alpha =}\NormalTok{ .}\DecValTok{5}\NormalTok{, }\DataTypeTok{col =} \StringTok{"black"}\NormalTok{) }\OperatorTok{+}\StringTok{ }
\StringTok{  }\KeywordTok{geom_point}\NormalTok{(}\KeywordTok{aes}\NormalTok{(}\DataTypeTok{x =}\NormalTok{ greenThumbDataset}\OperatorTok{$}\NormalTok{Longitude, }\DataTypeTok{y =}\NormalTok{ greenThumbDataset}\OperatorTok{$}\NormalTok{Latitude), }\DataTypeTok{data =}\NormalTok{ greenThumbDataset, }\DataTypeTok{alpha =} \DecValTok{1}\NormalTok{, }\DataTypeTok{col =} \StringTok{"purple"}\NormalTok{) }\OperatorTok{+}
\StringTok{  }\KeywordTok{theme}\NormalTok{(}\DataTypeTok{axis.title.x =} \KeywordTok{element_blank}\NormalTok{(), }\DataTypeTok{axis.title.y =} \KeywordTok{element_blank}\NormalTok{())}

\NormalTok{mapPoints}
\end{Highlighting}
\end{Shaded}

\includegraphics{geoPhyTask_files/figure-latex/visualize3-1.pdf}

The two datasets overlapped in only a few areas. By overlaying the value
per square foot I see that the overlapping areas have lower costs of
condomoniums.

\begin{Shaded}
\begin{Highlighting}[]
\NormalTok{mapPointsValue <-}\StringTok{ }\KeywordTok{ggmap}\NormalTok{(mapNY) }\OperatorTok{+}\StringTok{ }
\StringTok{  }\KeywordTok{geom_point}\NormalTok{(}\KeywordTok{aes}\NormalTok{(}\DataTypeTok{x =}\NormalTok{ condoDataset}\OperatorTok{$}\NormalTok{Longitude, }\DataTypeTok{y =}\NormalTok{ condoDataset}\OperatorTok{$}\NormalTok{Latitude, }\DataTypeTok{color =}\NormalTok{ condoDataset}\OperatorTok{$}\NormalTok{MANHATTAN.CONDOMINIUM.PROPERTY.Market.Value.per.SqFt), }\DataTypeTok{data =}\NormalTok{ condoDataset, }\DataTypeTok{alpha =}\NormalTok{ .}\DecValTok{5}\NormalTok{) }\OperatorTok{+}\StringTok{ }
\StringTok{  }\KeywordTok{scale_color_gradient}\NormalTok{(}\DataTypeTok{high =} \StringTok{"red4"}\NormalTok{, }\DataTypeTok{low =} \StringTok{"coral"}\NormalTok{, }\DataTypeTok{name =} \StringTok{"ValuePerSqFt"}\NormalTok{) }\OperatorTok{+}
\StringTok{  }\KeywordTok{geom_point}\NormalTok{(}\KeywordTok{aes}\NormalTok{(}\DataTypeTok{x =}\NormalTok{ greenThumbDataset}\OperatorTok{$}\NormalTok{Longitude, }\DataTypeTok{y =}\NormalTok{ greenThumbDataset}\OperatorTok{$}\NormalTok{Latitude), }\DataTypeTok{data =}\NormalTok{ greenThumbDataset, }\DataTypeTok{alpha =} \DecValTok{1}\NormalTok{, }\DataTypeTok{col =} \StringTok{"purple"}\NormalTok{) }\OperatorTok{+}
\StringTok{  }\KeywordTok{theme}\NormalTok{(}\DataTypeTok{axis.title.x =} \KeywordTok{element_blank}\NormalTok{(), }\DataTypeTok{axis.title.y =} \KeywordTok{element_blank}\NormalTok{())}

\NormalTok{mapPointsValue}
\end{Highlighting}
\end{Shaded}

\includegraphics{geoPhyTask_files/figure-latex/visualize4-1.pdf}

\subsection{Task 2 - Expensive Neighbourhood Tabulation
Area}\label{task-2---expensive-neighbourhood-tabulation-area}

In order to look at the property values of the NTAs, I looked at the
average of the property values in each NTA in the data.

\begin{Shaded}
\begin{Highlighting}[]
\NormalTok{condoDataset}\OperatorTok{$}\NormalTok{NTA <-}\StringTok{ }\KeywordTok{gsub}\NormalTok{(}\StringTok{"[[:blank:]]+"}\NormalTok{, }\StringTok{""}\NormalTok{, condoDataset}\OperatorTok{$}\NormalTok{NTA)}
\NormalTok{allCondoNTA <-}\StringTok{ }\KeywordTok{table}\NormalTok{(condoDataset}\OperatorTok{$}\NormalTok{NTA)[}\KeywordTok{order}\NormalTok{(}\KeywordTok{table}\NormalTok{(condoDataset}\OperatorTok{$}\NormalTok{NTA), }\DataTypeTok{decreasing =}\NormalTok{ T)]}
\NormalTok{ntaCostDF <-}\StringTok{ }\KeywordTok{data.frame}\NormalTok{(condoDataset}\OperatorTok{$}\NormalTok{NTA, condoDataset}\OperatorTok{$}\NormalTok{MANHATTAN.CONDOMINIUM.PROPERTY.Market.Value.per.SqFt, }\DataTypeTok{stringsAsFactors =}\NormalTok{ F)}

\NormalTok{avgNTACostDF <-}\StringTok{ }\KeywordTok{data.frame}\NormalTok{(}\DataTypeTok{NTA =} \KeywordTok{names}\NormalTok{(allCondoNTA), }\DataTypeTok{AverageValue =} \KeywordTok{numeric}\NormalTok{(}\DataTypeTok{length =} \KeywordTok{length}\NormalTok{(allCondoNTA)), }\DataTypeTok{stringsAsFactors =}\NormalTok{ F)}
\ControlFlowTok{for}\NormalTok{ (i }\ControlFlowTok{in} \DecValTok{1}\OperatorTok{:}\KeywordTok{length}\NormalTok{(allCondoNTA)) \{}
\NormalTok{  tmpRows <-}\StringTok{ }\NormalTok{ntaCostDF[}\KeywordTok{grep}\NormalTok{(}\KeywordTok{names}\NormalTok{(allCondoNTA)[i], ntaCostDF}\OperatorTok{$}\NormalTok{condoDataset.NTA, }\DataTypeTok{fixed =}\NormalTok{ T), ]}
\NormalTok{  avgNTACostDF}\OperatorTok{$}\NormalTok{AverageValue[i] <-}\StringTok{ }\KeywordTok{mean}\NormalTok{(tmpRows[[}\DecValTok{2}\NormalTok{]])}
\NormalTok{\}}
\end{Highlighting}
\end{Shaded}

There are a total of 28 NTAs in the condomonium dataset, of which the
one called ``park-cemetery-etc-Manhattan'' had one property that had the
highest value per square foot.

\begin{Shaded}
\begin{Highlighting}[]
\NormalTok{avgNTACostDF[}\KeywordTok{which.max}\NormalTok{(avgNTACostDF}\OperatorTok{$}\NormalTok{AverageValue), ]}
\end{Highlighting}
\end{Shaded}

\begin{verbatim}
##                            NTA AverageValue
## 28 park-cemetery-etc-Manhattan          299
\end{verbatim}

Since this NTA only has one property listed there are not many
conclusions that can be made, although a visualisation did show that it
is right in the middle of Central Park.

\begin{Shaded}
\begin{Highlighting}[]
\NormalTok{priciestNTA <-}\StringTok{ }\NormalTok{avgNTACostDF}\OperatorTok{$}\NormalTok{NTA[}\KeywordTok{which.max}\NormalTok{(avgNTACostDF}\OperatorTok{$}\NormalTok{AverageValue)]}
\NormalTok{priciestNTAData <-}\StringTok{ }\NormalTok{condoDataset[}\KeywordTok{grep}\NormalTok{(priciestNTA, condoDataset}\OperatorTok{$}\NormalTok{NTA, }\DataTypeTok{fixed =}\NormalTok{ T), ]}

\NormalTok{mapPointsPriciest <-}\StringTok{ }\KeywordTok{ggmap}\NormalTok{(mapNY) }\OperatorTok{+}\StringTok{ }
\StringTok{  }\KeywordTok{geom_point}\NormalTok{(}\KeywordTok{aes}\NormalTok{(}\DataTypeTok{x =}\NormalTok{ condoDataset}\OperatorTok{$}\NormalTok{Longitude, }\DataTypeTok{y =}\NormalTok{ condoDataset}\OperatorTok{$}\NormalTok{Latitude), }\DataTypeTok{data =}\NormalTok{ condoDataset, }\DataTypeTok{alpha =}\NormalTok{ .}\DecValTok{5}\NormalTok{, }\DataTypeTok{col =} \StringTok{"black"}\NormalTok{) }\OperatorTok{+}\StringTok{ }
\StringTok{  }\KeywordTok{geom_point}\NormalTok{(}\KeywordTok{aes}\NormalTok{(}\DataTypeTok{x =}\NormalTok{ greenThumbDataset}\OperatorTok{$}\NormalTok{Longitude, }\DataTypeTok{y =}\NormalTok{ greenThumbDataset}\OperatorTok{$}\NormalTok{Latitude), }\DataTypeTok{data =}\NormalTok{ greenThumbDataset, }\DataTypeTok{alpha =}\NormalTok{ .}\DecValTok{5}\NormalTok{, }\DataTypeTok{col =} \StringTok{"purple"}\NormalTok{) }\OperatorTok{+}
\StringTok{  }\KeywordTok{geom_point}\NormalTok{(}\KeywordTok{aes}\NormalTok{(}\DataTypeTok{x =}\NormalTok{ priciestNTAData}\OperatorTok{$}\NormalTok{Longitude, }\DataTypeTok{y =}\NormalTok{ priciestNTAData}\OperatorTok{$}\NormalTok{Latitude), }\DataTypeTok{data =}\NormalTok{ priciestNTAData, }\DataTypeTok{alpha =}\NormalTok{ .}\DecValTok{5}\NormalTok{, }\DataTypeTok{col =} \StringTok{"red"}\NormalTok{)}

\NormalTok{mapPointsPriciest}
\end{Highlighting}
\end{Shaded}

\includegraphics{geoPhyTask_files/figure-latex/maximumValue3-1.pdf}

\subsection{Task 3a - Green Thumb
Data}\label{task-3a---green-thumb-data}

There are 18 NTAs from the condominium dataset that have community
gardens.

\begin{Shaded}
\begin{Highlighting}[]
\NormalTok{greenThumbDataset}\OperatorTok{$}\NormalTok{NTA <-}\StringTok{ }\KeywordTok{gsub}\NormalTok{(}\StringTok{"[[:blank:]]+"}\NormalTok{, }\StringTok{""}\NormalTok{,  greenThumbDataset}\OperatorTok{$}\NormalTok{NTA)}
\NormalTok{commonNTA <-}\StringTok{ }\KeywordTok{intersect}\NormalTok{(greenThumbDataset}\OperatorTok{$}\NormalTok{NTA, }\KeywordTok{names}\NormalTok{(allCondoNTA))}
\NormalTok{colours <-}\StringTok{ }\KeywordTok{ifelse}\NormalTok{(avgNTACostDF}\OperatorTok{$}\NormalTok{NTA }\OperatorTok\StringTok{ }\NormalTok{commonNTA, }\StringTok{"purple"}\NormalTok{, }\StringTok{"black"}\NormalTok{)}

\KeywordTok{plot}\NormalTok{(avgNTACostDF}\OperatorTok{$}\NormalTok{AverageValue, }\DataTypeTok{col =}\NormalTok{ colours, }\DataTypeTok{xlab =} \StringTok{"NTAs"}\NormalTok{, }\DataTypeTok{ylab =} \StringTok{"Value per square foot"}\NormalTok{)}
\end{Highlighting}
\end{Shaded}

\includegraphics{geoPhyTask_files/figure-latex/greenThumb1-1.pdf}

Based on the plots it seems to me that the properties in areas with
community gardens have a less than average value per square foot for New
York City.

\subsection{Task 3b - Rental income}\label{task-3b---rental-income}

In order to complete this task I took as the rental income the values in
the column ``MANHATTAN.CONDOMINIUM.PROPERTY.Gross.Income.per.SqFt''.
This column has values ranging from 0 to 67, I did find it odd that
there were 2 zero values but based on the meta data I decided this was
the best set of values to use. I compared the value and the rental
income with a plot and a correlation

\begin{Shaded}
\begin{Highlighting}[]
\NormalTok{valueIncome <-}\StringTok{ }\KeywordTok{data.frame}\NormalTok{(}\DataTypeTok{value =}\NormalTok{ condoDataset}\OperatorTok{$}\NormalTok{MANHATTAN.CONDOMINIUM.PROPERTY.Market.Value.per.SqFt, }
                          \DataTypeTok{income =}\NormalTok{ condoDataset}\OperatorTok{$}\NormalTok{MANHATTAN.CONDOMINIUM.PROPERTY.Gross.Income.per.SqFt)}
\NormalTok{valueIncome <-}\StringTok{ }\NormalTok{valueIncome[}\KeywordTok{order}\NormalTok{(valueIncome}\OperatorTok{$}\NormalTok{value), ]}

\KeywordTok{matplot}\NormalTok{((valueIncome), }\DataTypeTok{type =} \StringTok{"l"}\NormalTok{, }\DataTypeTok{lty =} \DecValTok{1}\NormalTok{)}
\end{Highlighting}
\end{Shaded}

\includegraphics{geoPhyTask_files/figure-latex/rentalIncome1-1.pdf}

Except for very few points the two vectors follow the same trend accross
properties.

\begin{Shaded}
\begin{Highlighting}[]
\KeywordTok{cor}\NormalTok{(valueIncome}\OperatorTok{$}\NormalTok{value, valueIncome}\OperatorTok{$}\NormalTok{income)}
\end{Highlighting}
\end{Shaded}

\begin{verbatim}
## [1] 0.9753242
\end{verbatim}

The extremely high correlation value also shows that the rental income
could be an important determinant of property value.


\end{document}
